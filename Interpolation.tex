\documentclass[12pt,a4paper]{scrartcl}
\usepackage[utf8]{inputenc}
\usepackage[english,russian]{babel}
\usepackage{indentfirst}
\usepackage{misccorr}
\usepackage{graphicx}
\usepackage{amsmath}
\usepackage{bm}
\usepackage{listings}
\usepackage{graphicx}
\usepackage[normalem]{ulem}
\DeclareGraphicsExtensions{.pdf,.png,.jpg}
\author{А. А. Галиуллин}
\lstset{
	language=Python,
	basicstyle=\ttfamily,
	columns=fullflexible,
	frame=single,
	breaklines=true,
}
\begin{document}
	
	\begin{center}
	\Large
	Интерполяция \\
	\end{center}
	\begin{center}
		\large
		Выполнил Галиуллин Арслан, 1 курс факультета физики, группа 171. \\
		1233550v@mail.ru \\
		Желаемая оценка - 10.
	\end{center}
	Задача - построить такую функцию, чтобы её значения в узлах интерполяции совпадали с данными значениями. \\
	Это, очевидно, можно сделать многими способами. Я сделал двумя - вручную многочленом Лагранжа и при помощи встроенной в библиотеку scipy функции interp1d. \\
	
	\section{Многочлен Лагранжа}
		$$L_N(x) = \sum_{k=0}^{N}c_k(x)*u_k, c_k(x) = \prod_{\begin{smallmatrix}i = 0\\i\neq л\end{smallmatrix}}^{N} \frac{x-x_i}{x_k-x_i}$$
		$u_k$ - значение в узле, $x_j$ - узел.
		\\ \\
		\lstinputlisting{Interpolation_Lagrange.py}
		
		Вот пример нескольких функций:\\
		Зелёные точки - данные значения, синяя линия - точная функция (из которой случайным образом брались значения в узлах), красная линия - полином Лагранжа.\\
		\begin{center}
			\includegraphics[scale=0.8]{figure_1} \\
		\end{center}
		 $\sigma = 3.6 \cdot 10^{-5}$ - среднеквадратичное отклонение. \\
		\begin{center}
			\includegraphics[scale=0.8]{figure_2} \\
		\end{center}
		$\sigma = 0.12$ \\
		\begin{center}
			\includegraphics[scale=0.8]{figure_3} \\
		\end{center}
		$\sigma = 0.24$
		\begin{center}
			\includegraphics[scale=0.8]{figure_4} \\
		\end{center}
		$\sigma = 1.3 \cdot 10^{-6}$
	\section{Встроенная функция}
	\lstinputlisting{Interpolation_System.py}
	Тут точки соединяются многочленами третьей степени. \\
	Функции: \\
		\begin{center}
			\includegraphics[scale=0.8]{figure_5} \\
		\end{center}
		$\sigma = 0.12$ \\
		Виден недостаток этого метода: он плохо находит экстремумы. \\
		\begin{center}
			\includegraphics[scale=0.8]{figure_6} \\
		\end{center}
		$\sigma = 23$ - и погрешность больше. \\
		
		\section{Итог}
			Без какой-нибудь практической задачи кажется, что интерполяция многочленом Лагранжа лучше встроенной функции.
\end{document}